% perhaps put introduction in the front matter
% and have three chapters of 1) current understanding
%                            2) experimental considerations
%                            3) theoretical motivation
% decide later, do the writing first
\chapter{Introduction}
%\addcontentsline{toc}{chapter}{Introduction}
\thispagestyle{plainbottom}
Most children suffer from different kinds of developmental disorders. Among the many disorders present, the most common ones are ASD, ADHD and VCFS. While VCFS is a genetic defect, the causes of ASD and ADHD are not known. There various studies in this field to identify the genetics behind these two disorders. Also, few children suffer from more than one disorder and sometimes one leads to another. Therefore, it is essential to understand as much about these developmental disorders as possible.

Clinicians take almost 13 months to diagnose children and even later in the case of comorbid disorders. Many researchers have developed techniques for early intervention of these disorders. As these disorders effect the trajectory of growth of children, early intervention is critical for children and families to led normal lives. A lot of researchers believe that machine learning could speed up this process and play a vital role in early diagnosis. So, the main aim of our research is to analyze these disorders, understand them better and try to build models for prediction of these disorders. Our key contributions are as follows-
\begin{itemize}
\item Identifying important features for developmental disorders ASD, VCFS and ADHD, along with the comorbid disorders.
\item Assessing the impact of the three different parent-oriented reviews on these developmental disorders and their comorbidity.
\item Designing models to diagnose these different subgroups of developmental disorders present in the data
\item Designing models to diagnose for the comorbid disorders `ASD and VCFS' and `ASD and ADHD'
\item Designing models for each individual diagnosis that is ASD, VCFS and ADHD.
\item Formulating hypothesis using the features relation found in our models to assess the developmental disorders
\end{itemize}


Initially, in our thesis, more information is provided on these developmental disorders and their comorbidity. The parent-oriented reviews which will be used for our data are explained. Also, in chapter 2, the various models that have been designed using machine learning to diagnose these disorders are explained. Before applying machine learning on our data, in chapter 3, the data is analyzed and preprocessed. Furthermore, analysis and observations are made on the significant features related to these developmental disorders. Later on in chapter 4, understanding of various subgroups in our data is done using both unsupervised and supervised techniques. Additional information on the comorbidity of the disorders is examined in chapter 5 and finally, in chapter 6, each of these individual developmental disorders are investigated. 


