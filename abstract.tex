\begin{center}
\tracked{ABSTRACT}\\[12pt]
%\updateme
\end{center}
\vspace*{2cm}
\linespread{2}
%\addcontentsline{toc}{chapter}{Abstract}
Early developmental disorders are common in children between the ages of 3 through 17. These developmental disorders begin at early ages and affect the day-to-day activity of children. They also impact the growth and lifestyle of children. Most of the time these developmental disorders co-exist in children. The main focus of our research lies in the disorders `Autism Spectrum Disorder', `Attention-Deficit/Hyperactivity Disorder', `Deletion syndrome (22q)' and their co-occurrences.

Most child psychologists and pediatricians diagnose these disorders in children through parent-oriented reviews. Our research uses three different parent-oriented reviews, which are `Autism Diagnostic Interview', `Behavioral Assessment Schedule for Children', and `Vineland Adaptive Behavior Scales'. These reviews are questionnaires that parents answer under the inspection of certified professionals. While these examinations take up lots of time and yield results after at least 13 months of wait time, the process needs to speed up and hence, machine learning could play a vital role in this process.

Machine learning, when applied to reviews, can help understand the relevance and importance of them in diagnosing the disorders. Also, many techniques have been applied in our research to evaluate the co-occurrence of these disorders. The objective is to determine if machine learning could predict the occurrence of these disorders. Also, decide the significance of these reviews using feature selection algorithms of machine learning.
\setcounter{page}{1}
\thispagestyle{empty}
\pagestyle{empty}
\cleardoublepage
