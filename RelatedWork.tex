\linespread{2}
\chapter{Related Work}
%\addcontentsline{toc}{chapter}{Introduction}
\thispagestyle{plainbottom}
Developmental Disorders comprise of a group of psychiatric conditions that originate in childhood and cause severe problems in certain areas. According to the Centers for Disease Control and Prevention (CDC), more than one out of every 100 school children in the United States has some form of mental retardation. The third most common disorder is Autism Spectrum Disorder.
\nomenclature{ASD}{Autism Spectrum Disorder}Developmental disorders are caused because of various reasons like genetics (Deletion syndrome) or due to complications in pregnancy or birth. However, most of the times the exact causes are not known. It can also be seen that developmental disorders are common occurrences in children when the mother goes through stress or consumes alcohol during pregnancy~\cite{webpage2}. For proper livelihood of the child, early identification of these disorders is crucial as their causes could be reversed. Most children who were diagnosed with ASD usually reported initial signs of concern before the child was 2 years old in their records. However, most of these children were diagnosed only after age 4, almost 82\% of the time by age 3 and 21\% of the time at 8 years ages\cite{christensen2016prevalence}. Early identification helps families to prepare strategies and other resources for successful families. 

The occurrence of multiple disorders causes more complexity in identifying them as one of those disorders would be diagnosed and the other would take more time to identify. Children who are diagnosed with ASD could also be suffering from ADHD\nomenclature{ADHD}{Attention Deficit/Hyperactivity Disorder} or vice-versa. The relationship between these two disorders is still debated in the field of psychology. While children suffering from VCFS\nomenclature{VCFS}{22q Deletion Syndrome} could also be affected with Schizophrenia. It was observed that 2 in every 178 patients suffering from Schizophrenia had VCFS~\cite{bassett199922q11}. Also, a study done with data collected from a congenital cardiac clinic reported that 22.6\% of adults with VCFS had schizophrenia or schizoaffective disorder~\cite{bassett2005clinical}. 

Machine learning could help build models to analyze these causes much faster and help in speeding up the diagnosis process. With this research, we hope to identify the causes for single or multiple disorders with our models with good accuracies and also differentiate subjects based on different features. While machine learning algorithms seem to be showing promising results, there could be some pitfalls. It is essential that when applying these techniques the researchers should be made aware of all aspects of the study as this is an interdisciplinary research and some of these issues like methodology and implementation are discussed by author Wall ~\cite{wall2012use}. Further details about the relationship between these diseases are essential for our knowledge and hence, will be discussed before understanding how machine learning is playing a crucial role.
\section{Developmental Disorders and Comorbidity}
While most developmental disorders deal with mental retardation, Autism Spectrum Disorder(ASD) is characterized by social interaction difficulties and communication challenges. There are many types of autism which are caused by different genetic combinations and environmental influences. The Centers for Disease Control and Prevention (CDC) estimates that it is present in 1 out of 68 children in the United States. It seems to occur 4.5 times more frequently in boys than girls. The four sub-types of Autism are Autistic Disorder, Asperger’s Syndrome, Childhood Disintegrative Disorder and Pervasive Developmental Disorder according to the fourth edition of Diagnostic and Statistical Manual of Mental Disorders(DSM) which is published by American Psychiatric Association(APA). However, in the fifth edition published in 2013, all of these four sub-types were dissolved into a single disorder called Autism Spectrum Disorder and the American Psychiatric Association(APA) felt that it this would be helpful for accurately diagnosing individuals with autism~\cite{edition2013diagnostic}. 

Attention-Deficit/hyperactivity disorder(ADHD) is also a brain disorder with patterns of inattention and/or hyperactivity that affects the development of a child. Children who suffer from ADHD have difficulty in staying focused and paying attention, and difficulties with controlling behavior. Similar to ASD, ADHD also seems to have more prevalence among male children~\cite{visser2014trends}. This disorder also continues to affect the children in their adulthood, almost one-third of children diagnosed with ADHD retain it even in adulthood~\cite{barbaresi2013mortality}. According to the fourth edition of Diagnostic and Statistical Manual of Mental Disorders, published in 2000 by APA, there are three sub-types of ADHD. The three sub-types are combined type of ADHD, predominantly inattentive type ADHD and predominantly hyperactive-impulsive type ADHD. Both ASD and ADHD are disorders for which the causes are not clear and also researchers are not sure about the role played by environmental factors.

Deletion syndrome 22q11.2(VCFS) is also known as DiGeorgesyndrome as it was first described by Angelo DiGeorge in 1968~\cite{digeorge1968congenital,restivo200622q11}. It is caused by the deletion of 30 to 40 genes in the middle of chromosome 22 at a location known as 22q11.2. This is a genetic disorder and 10\% of the cases are inherited by a parent. DiGeorge syndrome occurs in about 1 in 4,000 people. Children diagnosed with this disease have delayed growth and speech development, and also have learning disabilities. Individuals affected by this disorder have breathing problems, low levels of calcium in blood and kidney abnormalities. Unlike ASD and ADHD, VCFS can be diagnosed by performing genetic analysis to detect microdeletions. Fluorescencein situhybridization(FISH)is a method which helps diagnose VCFS, quantitative polymerase chain reaction(qPCR) is quicker than FISH. It has a turn around time of 3 to 14 days. Those affected by VCFS have the risk of developing other disorders like schizophrenia, depression, anxiety, and bipolar disorder~\cite{mcdonald2011chromosome}.

While each of these disorders affects a person in a different way they are also related to one another or more specifically one leads to another. When these disorders co-occur then they are called comorbid disorders. As mentioned earlier ASD and ADHD which co-occur are comorbid disorders while VCFS and schizophrenia are comorbid disorders. It is essential to understand their comorbidity to better analyze and diagnose these disorders.
\subsection{Co-occurrence of VCFS and Schizophrenia}
Schizophrenia is also a mental disorder where the diagnosed fail to understand reality and lack of normal social behavior. People with schizophrenia have unclear thoughts, anxiety, depression and reduced social engagement~\cite{cross2017external} ~\cite{webpage1}. The evidence for VCFS being the first identifiable genetic subtype of schizophrenia was given by Anne. S. Bassett~\cite{bassett2008schizophrenia}. In this research, it was found that individuals whose relatives have schizophrenia tend to have a lower probability of containing individuals with 22qDS. Most studies that were done with different samples of data show that the relative risk for schizophrenia in an individual initially diagnosed with 22qDS is about 20 to 25 times the lifetime general population risk of 1\% ~\cite{bassett2000chromosomal}~\cite{hodgkinson2001genetic}.

When a study was conducted on non-overlapping samples, in which 82 individuals had VCFS and schizophrenia showed some comparable clinical signs~\cite{bassett2008schizophrenia}. By comparing the IQ of the patients, some reports showed that the IQ was same that is ranged from 70-84 irrespective of whether the patient had schizophrenia or not~\cite{chow2006neurocognitive}. On the other hand, another study found mental retardation in 69\% patients, found lower IQ levels for patients with VCFS and schizophrenia than patients with no schizophrenia\cite{raux2006involvement}. Some other findings that distinguish these patients were that they had smaller brain volume, midline defects such as cavum septum pellucidum, and white matter hyper intensities on MRI~\cite{van2004brain}~\cite{chow1999qualitative}. 

Due to these studies being carried out even though most of the times it continues to be unrecognized, clinicians are educated of the possibilities during diagnosing individuals. This would help them look for signs of schizophrenia in patients with VCFS. Also, they could let patients known about possibilities of schizophrenia and other disorders too.
\subsection{Co-occurrence of VCFS and ASD}
According to research, 15-20\% patients diagnosed with VCFS meet the behavioral criteria for a diagnosis of ASD. Due to the missing DNA of the 22 chromosome,  there could be modifying affects in each person’s set of genes. Some individuals could have social difficulties, developmental delays or learning disabilities. These are some of the symptoms of an autistic child. Hence, it is believed that understanding genetic causes for autism is important.

Various studies have been conducted to examine the extent to which VCFS individuals also have ASD. One of these studies uses children, adolescents and young adults whose males to females ratio is 2.3:1. Based on the information from the Autism Diagnostic Interview-Revised (ADI-R), the Autism Diagnostic Observational Schedule (ADOS), and a clinician’s best-estimate diagnosis, they estimated that ASD is present in 15-50\% of the cases~\cite{ousley2017examining}.

By studying the phenotypes, differences with the children with VCFS and VCFS+ASD is being researched. From this analysis it was seen that 94\% of the children with VCFS + ASD had a co-occurring psychiatric disorder, on the other hand, 60\% of children with VCFS had a psychiatric disorder. Also, further studies of the brain showed that children with VCFS + ASD had larger right amygdala volumes and all other neuro-anatomic regions of interest were statistically similar between the two groups~\cite{antshel2007autistic}.

Apart from schizophrenia, ASD is also one of those psychiatric disorders that children with VCFS could possibly have. In our study, some analysis has been done to study the phenotypes of children with VCFS and VCFS+ASD using machine learning techniques. Techniques have mainly been applied to find distinguishable features and help broaden the understanding with the co-occurrence of these two developmental disorders.

\subsection{Co-occurrence of ASD and ADHD}
ASD and ADHD are two developmental disorders that affect an increasing number of children. These two developmental disorders share some common symptoms like lack of concern, not being able to react to others emotions or feelings etc. The reason why it is difficult to distinguish symptoms of ASD from ADHD is that they occur at the same time. In one of the studies in pediatrics field, showed that 18\% of the children diagnosed with ADHD showed signs of ASD~\cite{kotte2013autistic}. So, when this disorder occurs in children, they tend to have learning complications and impaired social skills.

Various studies have been done in the field of psychology to understand the relation between these two developmental disorders. However, researchers are still not certain about why these two developmental disorders occur together frequently. Initially, doctors never diagnosed a child with these two disorders, as they believed that they could not coexist as per the APA. It was in 2013, the release of the Diagnostic and Statistical Manual of Mental Disorders, Fifth Edition (DSM-5) when the APA stated that ASD and ADHD could co-occur~\cite{american2013diagnostic}. Researchers found that 20-50\% of the individuals diagnosed with ADHD had shown symptoms of ASD as well and 30-80\% of the individuals diagnosed with ASD meet the criteria for ADHD~\cite{rommelse2010shared}. It is also believed that when these two disorders co-occur, they cause greater morbidity and create a more complicated clinical challenge.

As the research in this domain is more recent, there is two possible hypothesis of these two disorders co-occurring. The first hypothesis is that ASD and ADHD are distinct yet overlapping diseases which may share some common etiology. The second hypothesis is that ASD and ADHD co-occurrence “stands alone” as a distinct clinical disorder, with a distinct etiology~\cite{leitner2014co}. The author Nat Gene tried to find a genetic relationship between five different psychiatric disorders which included ADHD and ASD using genome-wide SNPs, and according to his research, there seems to be a non-significant genetic correlation between the two disorders (ASD and ADHD)~\cite{lee2013genetic}. 

A few researchers from the Netherlands have suggested that ASD and ADHD are different manifestations of a single medical condition with different subtypes. As per their research, ADHD can occur independently without signs of ASD, but ASD always occurs with symptoms of ADHD\cite{van2012autism}. When examining the brain images of persons with ASD, ADHD and both, the analysis showed shared and distinct brain alterations. For patients with both the disorders, there was an overlap in the corpus callosum and cerebellum (lower volume in structural MRI and decreased FA in DTI), and superior longitudinal fasciculus. They also found that the corpus callosum and cerebellum are usually smaller than usual size~\cite{dougherty2016comparison}.On the other hand, few researchers examined the brain images as per the second hypothesis and concluded that brain maturationin both conditions proceeds differently or is delayed for individuals with ASD and ADHD. They also believe that distinct patterns of thinning in definite brain images will help them define subtypes of the ASD-ADHD syndrome~\cite{rommelse2017structural}.

As there are varied researches related to the co-occurrence of these two disorders, another approach believes on focusing on the traits(behavior) rather than genetic or brain features diagnosis~\cite{Kottepeds.2012-3947}. Similarly, in our research, we try to examine the relations with the phenotypes of ASD, ADHD and both. So, different machine learning techniques have been used to draw conclusions about their co-occurrence.

\section{Parent-oriented Reviews for developmental disorders}
These are reviews in the form of questions which parents answer. For developmental disorders, parents are given a set of different questions to access the conditions of the children and diagnose various disorders. Parents play a vital role when diagnosing children for developmental disorders, as they are the ones who seem to observe most of the signs and symptoms like when the child starts to speak when a child distracted etc. There are many parent-oriented review, but this section, three parent-oriented reviews will be explained in detail. The three parent-oriented reviews are Autism Diagnostic Interview (ADI)\nomenclature{ADI}{Autism Diagnostic Interview}, Behavioral Assessment Schedule for Children (BASC)\nomenclature{BASC}{Behavioral Assessment Schedule for Children} and Vineland Adaptive Behavior Scales (VINE)\nomenclature{VINE}{Vineland Adaptive Behavior Scales}.

\subsection{Autism Diagnostic Interview (ADI)}
This is a semi-structured parent review to check for ASD related behaviors in a child. This test is performed to mainly analyze four domains in children. These four domains are Reciprocal Social Interaction, Language/Communication, Restricted, Repetitive, and Stereotyped Behaviors and  Abnormality Present in early development (before age 3). This review usually takes two hours and the parents are asked 93 different questions which are related to the above four domains. The scoring for these questions is 	0- “Behavior of the type specified in the coding is not present”, 1- “Behavior of the type specified is present in an abnormal form, but not sufficiently severe or frequent to meet the criteria for a 2, 2- "Definite abnormal behavior” and 3- "Extreme severity of the specified behavior”. A child is diagnosed with ASD when the scores exceed the minimum cutoff scores~\cite{bone2015applying}. These minimum cutoffs have been identified after many years of research.

Extensive research on the ADI shows that it is an extremely helpful mode of assessment and useful for treatment as well as education planning. The ADI was able to diagnose children with a chronological age of at least five years and a mental age of at least two years. Later on, in 1994, ADI was revised (ADI-R)\nomenclature{ADI-R}{Autism Diagnostic Interview Revised} to focus on the first three domains and the Abnormality Present in early development was removed as these features very less relevant compared to other domains\cite{lord1994autism}. The main advantage of ADI-R is that it means the DSM-IV criteria and is focused on children in the 3-5 years range and a mental age of 18 months. It also has adequate sensitivity and specificity when administered by highly-trained personnel. The extensive use of ADI-R in the international research community seems to provide strong evidence of the reliability and validity of its categorical results in diagnosing ASD. It is proven to be affective in distinguishing ASD from other developmental disorders and identifying new subgroups\cite{rutter2003autism}.

Even though research shows that ADI-R is a good method to diagnose children with ASD at an early age(3 years), there also are researchers whose research show examples of children meet the criteria of ADI-R but don’t have ASD. One such example is given where three children met research criteria for an ASD by meeting or exceeding the cut-off scores in the communication and social interaction domains, however, their social and communication behaviors were not similar to those of an autistic child. Also, when a clinical psychologist reviewed the children, they were not diagnosed with ASD~\cite{reaven2008use}. So, it can be seen that there are chances of having false positives results with these tests.

\subsection{Behavioral Assessment Schedule for Children(BASC)}
Behavioral Assessment Schedule for Children is mainly designed to evaluate children with psychological problems. In 2004, BASC was revised to BASC-2 which helped to evaluate behavioral and personality aspects which included positive(Adaptive) along with negative(Clinical) dimensions~\cite{american2013diagnostic}. To improve the flexibility of administering and enhancing progressive monitoring of children with emotional disabilities, BASC 2 was further revised to BASC 3 in 2015.

The main purpose of this review is assessing emotional/behavioral disorders in children and adolescents. The set of rating sales of BASC are comprehensive that help in evaluating child behaviors from a different perspective like a parent, teacher and so on. The different scales and forms of BASC are Teacher Rating Scales (TRS), Parent Rating Scales (PRS), Self-Report of Personality (SRP), Student Observation System (SOS), and Structured Developmental History (SDH)~\cite{american2013diagnostic}. Most of our research deals with the PRS and TRS which has the highest correlations with Hyperactivity, Aggression, Atypicality, Withdrawal, and Attention Problems~\cite{american2013diagnostic}.

The SRP, the TRS, and the PRS are scored with T-scores that depend on a national norm group, by gender in the norm group, or in comparison to clinical population. However, the SDH and the SOS do not have specific norms~\cite{american2013diagnostic}. The children having T-scores above 40 are considered `Average', T scores between 30-39 are considered `Borderline' or `At Risk' and T scores below 30 are considered `clinically significant'. 
\newline
\newline
In one of the studies, there were signs of elevated `Anxiety', `Atypicality' or `Social Withdrawal' scores for children with VCFS indicating risk for schizophrenia\cite{kates2015neurocognitive}. Most of the time BASC has been applied to diagnose ASD and ADHD over VCFS. When diagnosing attention deficit disorders in children BASC, it has proven to perform more accurately then Child Behavior Checklist (CBCL). Also, it was able to explain salient behavioral dimensionsrelated to various ADHD subtypes~\cite{ostrander1998diagnosing}. BASC has also shown that children diagnosed with ADHD are rated lower on adaptive skills when compared to children with no diagnosis~\cite{jarratt2005assessment}. However, when trying to diagnose children with ASD, atypical behavior, attention and adaptive functions were complicated. It was also observed that the parent-rated social withdrawal was higher for children with ASD~\cite{gardner2017comparing}.

\subsection{Vineland Adaptive Behavior Scales (VINE)}
Vineland Adaptive Behavior Scales which one of the many assessment tools available for students with special needs. It was developed by three social research scientists Sara Sparrow, David Balla, and Domenic Cicchett~\cite{sparrow1984vineland}. It is used to measure adaptive behaviors, including the ability to cope with environmental changes, to learn new everyday skills. The revised version of the Vineland Adaptive Behavior Scales was made in 2005, to better measure adaptive skills in very young children and to capture qualitative differences in communication and social interaction for individuals on the autism spectrum.

Researchers have found that when diagnosing children with ADHD who had average full-scale IQs\nomenclature{IQ}{Intelligent Quotient}, they had Vineland standard scores in the borderline to low-average range. It was also observed that ADHD children with tertiary attention problem had significant social adaptive dysfunction on the Vineland~\cite{roizen1994adaptive}. On the other hand, when analyzing the adaptive behavior of children with ASD using VINE, low scores were found in social skills while high scores were found in motor skills~\cite{yang2016vineland}. Also, when distinguishing the ASD+ADHD children from the autistic children, the prior had lower scores on the VINE and the Pediatric Quality of Life Inventory. It was found that autistic children had greater impairment in adaptive functioning and clinically significant in children who suffered from both ASD and ADHD. However, children who suffered from only ADHD had fewer symptoms~\cite{sikora2012attention}. Furthermore, when children with VCFS were studied it could be seen that boys diagnosed with VCFS scored lower than girls diagnosed with VCFS on communication, daily living skills and socialization measures of VINE. The study also found that there was a negative correlation between age and cognitive function with girls, that is the scores did not keep up with expected improvement with age, while it was not the case for boys~\cite{antshel200522q11}.

All the above example show that these parent-oriented reviews are good indicators of the three developmental disorder ASD, ADHD, and VCFS. They are reliable and used by most psychologists or clinicians to diagnose children. For our research, the variables/features are these three parent-oriented reviews results for subjects which will be used for our analysis.

\section{Machine Learning applied to assess developmental disorders}
Machine learning is a field of computer science wherein the computers have the ability to learn from data rather than being explicitly programmed~\cite{samuel1959some}.\nomenclature{ML}{Machine Learning} The term machine learning was given by an American pioneer Arthur Samuel in 1959~\cite{kohavi1998glossary}. It was not until the 1990s that machine learning started to flourish as a separate field. While psychology and machine learning seem to be two independent studies, the field which is a combination of both these fields is Cognitive Science. Apart from these two fields, there are researchers from other fields like biology, neurosciences, sociology and so on who contribute towards this field. There are different research methods used  in cognitive sciences which are derived from different fields. However, behavioral experiments and brain imaging are two methods related to developmental disorders and will be discussed in greater details.
\subsection{Behavioral Experiments}
These are experiments to measure behavioral responses to different stimuli and understand about how those stimuli proceed. The measures used are of three types- behavioral traces, behavioral observations, and behavioral choice ~\cite{lewandowski2009actions}. Behavioral traces are pieces of evidence that indicate behavior occurred, but the actor causing the behavior is not present. On the other hand, behavioral observations involve the direct witnessing of the actor engaging in the behavior and behavioral choices are when a person selects between two or more options.

The reaction time of the person can indicate differences between cognitive process. This could be used to analyze different things in individuals and draw various conclusions. It was proven that reaction times are highly correlated to intelligence, which means that highly intelligent people tend to process speed faster. So, it could be seen that reaction times could be indicators of psychological distress ~\cite{deary2001reaction}. Similarly, in this study, the authors try to differentiate ASD patients(18) and ADHD patients(30) from control groups(13) ~\cite{uluyagmur2016adhd}. The basis for their differentiation is that patients with developmental disorders like ASD and ADHD fail to understand certain emotions. Also, it was observed that children with ASD have shorter reaction times when compared to those that do not have ASD ~\cite{baisch2017reaction}. So, the data for this study was collected based on the children’s ability to understand emotions. The features are related to the response and response latency of the children. The children were grouped into seven different groups and the based on the responses features were created. They applied ReliefF feature selection algorithm and the machine learning algorithms applied are Decision Tree, Random Forest, Support Vector Machine, K-nearest neighbor and Ada Boost. Their main aim was to study the emotional status of the patients and based on this identify their diagnosis. The overall study shows that ASD children could be differentiated from ADHD and control group with 80\% accuracy ~\cite{uluyagmur2016adhd}.

\subsection{Brain Imaging}
The analysis of activities in the brain while performing various tasks is called brain imaging. By linking the brain function and behavior, we can understand how information is processed. As discussed earlier, brain images of children with developmental disorders are studied, these help us during diagnosis.

The functional magnetic resonance imaging(fMRI) measures brain activity by detecting changes associated with blood flow ~\cite{rinck2014magnetic}. As some researchers have used deep learning techniques to study the fMRIs of children suffering from autism and tried to find patterns that could differentiate them from control groups. There are two widely known datasets Autism Brain Imaging Data Exchange(ABIDE) and ADHD-200 which have fMRI brain images of children who are diagnosed with ASD and ADHD respectively. In 2016, researchers applied a learning technique called `(f)MRI HOG-feature-based patient classification(MHPC)' on these datasets that uses the Histogram of oriented gradients (HOG) features to predict ADHD and ASD from their respective datasets. This algorithm was able to achieve a hold-out accuracy of 69.6\% for distinguishing ADHD and hold-out accuracy of 65.0\% for distinguishing ASD from the control groups ~\cite{ghiassian2016using}.

Recently in 2017, researchers improved the state-of-art model by achieving 70\% accuracy in identifying ASD patients from control groups. The data used for this research is also the brain imaging data from a world-wide multi-site database known as Autism Brain Imaging Data Exchange(ABIDE)\nomenclature{ABIDE}{Autism Brain Imaging Data Exchange}. A connectivity matrix was built using correlation which is calculated for the average of the time series of the regions of interest. Then different classification algorithms like Support Vector Machines, Random Forest, and Deep Neural Nets was applied for the purpose of prediction. The DNN achieved a mean accuracy of 70\% with 74\% sensitivity and 63\% specificity, while the SVM and RF model achieved a mean accuracy of 65\% and 63\% respectively. Apart from prediction, the researchers applied machine learning techniques to identify different areas of the brain that are negatively correlated and positively correlated to ASD. Even though their research is not unto to the biomarker standards, they believe that further research can be very helpful for developmental disorders diagnosis ~\cite{heinsfeld2018identification}.

Support vector machine was also applied to imaging data of patients with VCFS. When applying diffusion imaging methods, white matter micro-structural abnormalities identified have affected a variety of neuro-anatomical tracts in 22q11.2DS. So, applying SVM on these diffused images can help optimize the selection process to discriminate VCFS patients from others. The mean accuracy obtained on the validation set was 84.8\% and also the researchers were able to identify important diffusion features in the imaging data ~\cite{tylee2017machine}.

So, these examples show that machine learning techniques have aided psychological methods. They have helped provide more insights and even though the results do not seem to result in clear real-world usage, they definitely show signs for more valuable research in this field.

\subsection{Applying Machine Learning on Screening processes}
Most of these parented-oriented reviews are time-consuming and there are lots of different reviews that exist which aid in diagnosing children with developmental disorders. For diagnosing children with ASD, there are reviews like ADI, ADI-R, ADOS and so on. Also similarly for diagnosing children with ADHD, there reviews like BASC, ADI, VINE, CBCL and so on. The method of diagnosing is based on the clinician or psychologist who is assessing the children. While each of these reviews have their own advantages and disadvantages, the common problem with them is that they consume a lot of time. So, researchers believe that applying machine learning techniques like feature selections can help speed up the diagnosis process and also aid in better understanding of these reviews.

Feature selection is also known as variable selection is the process of selecting a subset of relevant features for our model construction ~\cite{james2013introduction}. Researchers apply feature selection algorithm to Social Responsiveness Scale(SRS) score sheets of individuals who either had ASD or ADHD. There are 65 questions in the SRS and using forward feature selection algorithm called minimal-redundancy-maximal-relevance (mRMR) criterion, they selected top 6 features. This feature selection algorithm tends to select features with a high correlation with the class (output) and a low correlation between themselves. Then they applied four different machine learning algorithms from the scikit-learn package\nomenclature{sklearn}{scikit-learn package} which are SVC, LDA, Categorical Lasso and Logistic Regression. The observed that the comparable accuracies to be 0.962 - 0.965 and plotted the ROC. This shows that these features are doing a great job of distinguishing ASD patients from ADHD patients ~\cite{duda2017use}. 

Other researchers applied the ADTree machine learning algorithm on the Autism Diagnostic Observation Schedule-Generic(ADOS) to evaluate ASD. This behavioral evaluation consists of four different modules and each module takes around 30 and 60 minutes to deliver. When applying this technique to the module 1 of ADOS, it showed that 8 of the 29 questions are relevant. After training the classifier on these eight items, it achieved 99.7\% sensitivity and 94\% specificity ~\cite{wall2012use}. So, there model is a tree-based approximation that classifies subjects into the ASD and non-ASD groups.

From the above examples it can be seen that even though machine learning has been applied to diagnose developmental disorders recently, it has shown promising results. Hence, for our research, different machine learning techniques have been applied on combinations of data, to contribute to this research to some extent. Some of these machine learning techniques will help us build models like the supervised and unsupervised learning techniques, while other machine learning algorithms like feature selection will help is analysis of the data. Therefore, different kinds of analysis has resulted in conclusions that can contribute to understanding these developmental disorders from a different perspective.
